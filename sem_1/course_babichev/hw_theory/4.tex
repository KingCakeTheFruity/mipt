\documentclass{article}
\usepackage{hyperref}
\usepackage[english,russian]{babel}
\usepackage[a4paper, total={6in, 8in}]{geometry}
\begin{document}
\large

\begin{flushleft}
Горишний Максим, Б05-033, ДЗ 4
\end{flushleft}

\begin{center}
\huge №1
\end{center}

Это копия решения из 3го задания. На самом деле, я уже сам сомневаюсь, что ассимптотика выходит верной, она либо как раз $O(n + n\log n)$, либо $O(n + n \log^2n)$. Не понимаю, как посчитать, что перевесит - то, что каждый элемент изменит своё мн-во $\log n$ раз или то, что и само изменение произойдёт за $O(\log n)$, что в итоге даст $O(\log^2n)$

Изначально создадим n ДД - по одному размера 1 на каждую вершину -> $O(n)$. По запросу слияния будем из меньшего ДД будем явно добавлять эл-ты (по одному) в больший, при этом в отдельном массиве храня указатель на ДД, которому принадлежит этот элемент. Таким образом каждый элемент будет добавлен в большее дерево не более $\log n$ раз (т.к. при каждом добавлении размер его собственного дерева будет увеличиваться минимум вдвое), и т.к. всего элементов n, нам потребуется не более $n\log n$, чтобы слить все элементы в одно множество, значит, имеющаяся ассимптотика - $O(n + n\log n)$. Однако тут надо понять, что число запросов мерджа не превосходит $q$, значит, значит больший вклад в ассимптотику внесет не $n\log n$, а $q\log n$. Строго говоря, выделим отдельно $O(n)$ запросов на слияние и заппишем теперь $O$:
$O(n\log n + (q - n)\log n) = O(n\log n + (q - n)\log n) = O(q\log n)$
Если же запросов слияния меньше n, то мы ещё на этапе слияния получаем qlogn - т.к. сливаем не больше q эл-тов.
(Запросы поиска всегда будем оценивать их среднем O(logn) в ДД из всех эл-тов)
Тогда итого получаем O(n + qlogn), что нам и требовалось

\begin{center}
\huge №2
\end{center}

Используем ДО, в котором будем хранить сумму эл-тов на отрезке, а так же сумму квадратов и кубов. Зная всё это нетрудно вывести формулу для каждого типа запроса: (за $S_{lr}$ обозначим ответ, получаемый из сумм первых степеней, $Q_{lr}$ - квадратов, $C_{lr}$ - кубов)

$$ 1) \sum_{l \leq i \leq r} a_i =  S_{lr} $$
$$ 2) \sum_{l \leq i < j \leq r} a_i \cdot a_j =  \frac{S_{lr}^2 - Q_{lr}}{2} $$
$$ 3) \sum_{l \leq i < j < k \leq r} a_i \cdot a_j \cdot a_k =  \frac{S_{lr}^3 - C_{lr} - 3 Q_{lr}S_{lr}}{6} $$

Даже если формулы сами по себе не верны, задача точно решается таким образом

\begin{center}
\huge №3
\end{center}

Решим для начала задачу при случае, когда в массиве только два типа чисел - 0 и 1. \textbf{and} тогда становится \textbf{min} на отрезке, а \textbf{xor} собой и остаётся, т.е. это обычная задача ДО, которую мы умеем решать: делаем отложенные операции и не забываем, что если из сыновей мы сейчас получили \textbf{and} равный 0, то \textbf{xor} с единицей оставит этот 0 нулём, а вот если 1, то он превратит её в 0.

Отлично, применим то, что мы только решили, 10 раз - по одному для каждого бита чисел массива $a$ в котором $0 \leq a_i \leq 2^{10}-1$. Константа, конечно, будет 10, но ассимптотику $O(n + q \log n)$ мы получаем.

\begin{center}
\huge №4
\end{center}

Устроим сканлайн по информации об уже купивших билетах пассажирах. По ходу дела будем хранить, внимание, персистентное дерево на сиденьях поезда (т.е. на $S$ вершинах), поддерживающее минимум. Один кусок информации о пассажире - это номего его место, станция захода $l$ и выхода $r$. Обрабатывая открытие отрезка $[l, r]$, в вершину, отвечающую за место, на которое пришёл пассажир, записываем $+\inf$. Обрабатывая закрытие отрезка (выход пассажира) - записываем в вершину его $l$. Новую ветку для перс. дерева мы создаем на каждой новой станции, на которой пассажиры входили или выходили.

Но что вообще у нас вышло?

На самом деле теперь мы для станции под некоторым номером, используя версию ДО, отвечающую за эту станцию, знаем минимальную станцию, с которой до нашей можно доехать без пересадок! Более того, т.к. сейчас наше ДО одновременно с этим представляет дерево поиска по сиденьям, то возможно обычным спуском получить минимальный номер сиденья, на котором мы может доехать с некоей конкретной станции до нашей. А это - как раз те самые запросы, которые у нас и спрашивают. (если же до этой станции с запрашиваемой доехать без пересадок нельзя, то мы узнаем об этом, т.к. знаем мин. станцию, с которой добраться можно не вставая с сиденья)

Асимптотика тут большая и страшная, в неё как минимум должны входить m версий перс. ДО, по которым мы делаем обновления за logS (т.к. строили его на S посадочных мест), плюс, похоже, по m потребуется сжатие координат -> mlogm, по которому мы к тому же спускаемся за log(s). Плюс имеется q запросов. Значит на построение + ответы у нас уйдет $O\left( (q + m) \cdot \log (m + 2 \cdot s) \right)$, что как раз равноценно искомой ассимптотике.

$= O\left( (q + m) * \log (m + s) \right)$


\end{document}